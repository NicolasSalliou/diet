\RequirePackage[l2tabu, orthodox]{nag}
\documentclass[version=3.21, pagesize, twoside=off, bibliography=totoc, DIV=calc, fontsize=12pt, a4paper]{scrartcl}
%TODO consider \usepackage{slantsc} and \AtBeginDocument{%   \DeclareFontShape{T1}{lmr}{m}{scit}{<->ssub*lmr/m/scsl}{}% }, https://tex.stackexchange.com/questions/32942/italic-shape-needed-in-small-caps-fonts  https://tex.stackexchange.com/questions/284338/italic-small-caps-not-working for the case: \ac{thing}, typeset in an Example environment, so that the small caps are italicized.
%Permits to copy eg x ⪰ y ⇔ v(x) ≥ v(y) from PDF to unicode data, and to search. From pdfTeX users manual. See https://tex.stackexchange.com/posts/comments/1203887.
	\input glyphtounicode
	\pdfgentounicode=1
%Latin Modern has more glyphs than Computer Modern, such as diacritical characters. fntguide commands to load the font before fontenc, to prevent default loading of cmr.
	\usepackage{lmodern}
%Encode resulting accented characters correctly in resulting PDF, permits copy from PDF.
	\usepackage[T1]{fontenc}
%UTF8 seems to be the default in recent TeX installations, but not all, see https://tex.stackexchange.com/a/370280.
	\usepackage[utf8]{inputenc}
%Provides \newunicodechar for easy definition of supplementary UTF8 characters such as → or ≤ for use in source code.
	\usepackage{newunicodechar}
%Text Companion fonts, much used together with CM-like fonts. Provides \texteuro and commands for text mode characters such as \textminus, \textrightarrow, \textlbrackdbl.
	\usepackage{textcomp}
%Solves bug in lmodern, https://tex.stackexchange.com/a/261188; probably useful only for unusually big font sizes; and probably better to use exscale instead. Note that the authors of exscale write against this trick.
	%\DeclareFontShape{OMX}{cmex}{m}{n}{
		%<-7.5> cmex7
		%<7.5-8.5> cmex8
		%<8.5-9.5> cmex9
		%<9.5-> cmex10
	%}{}
	%\SetSymbolFont{largesymbols}{normal}{OMX}{cmex}{m}{n}
%More symbols (such as \sum) available in bold version, see https://github.com/latex3/latex2e/issues/71.
	\DeclareFontShape{OMX}{cmex}{bx}{n}{%
	   <->sfixed*cmexb10%
	   }{}
	\SetSymbolFont{largesymbols}{bold}{OMX}{cmex}{bx}{n}
%There’s no bold small caps in Latin Modern, we switch to Computer Modern for bold small caps, see https://tex.stackexchange.com/a/22241
	%\normalfont\DeclareFontShape{T1}{lmr}{bx}{sc} { <-> ssub * cmr/bx/sc }{}
%Warn about missing characters.
	\tracinglostchars=2
%Nicer tables: provides \toprule, \midrule, \bottomrule.
	%\usepackage{booktabs}
%For new column type X which stretches; can be used together with booktabs, see https://tex.stackexchange.com/a/97137.
	%\usepackage{tabularx}
%math-mode version of "l" column type TODO might require \usepackage{array} for \newcolumntype macro
	%\newcolumntype{L}{>{$}l<{$}}; TODO see why simple example from https://fr.wikibooks.org/wiki/LaTeX/Tableaux#L'environnement_tabular*,_ma%C3%AEtrise_de_la_largeur_d'un_tableau fails with DJ poster.
%Provides \addtocmd, \patchcmd, \xpretocmd, \newtoggle commands. TODO may be false, \newtoggle provided by etoolbox… xpatch extends etoolbox. Also loads xparse, which provides \NewDocumentCommand and similar commands possibly intended as replacement of \newcommand in LaTeX3 (for package authors? see https://tex.stackexchange.com/q/98152 and https://github.com/latex3/latex2e/issues/89).
	\usepackage{xpatch}
%ntheorem doc says: “empheq provides an enhanced vertical placement of the endmarks”; must be loaded before ntheorem. Loads the mathtools package, which loads and fixes some bugs in amsmath and provides \DeclarePairedDelimiter. amsmath is considered a basic, mandatory package nowadays (Grätzer, More Math Into LaTeX).
	\usepackage[ntheorem]{empheq}
%Package frenchb asks to load natbib before babel-french. Package hyperref asks to load natbib before hyperref.
	\usepackage{natbib}

\newtoggle{LCpres}
	\newtoggle{LCart}
	\newtoggle{LCposter}
	\makeatletter
	\@ifclassloaded{beamer}{
		\toggletrue{LCpres}
		\togglefalse{LCart}
		\togglefalse{LCposter}
		\wlog{Presentation mode}
	}{
		\@ifclassloaded{tikzposter}{
			\toggletrue{LCposter}
			\togglefalse{LCpres}
			\togglefalse{LCart}
			\wlog{Poster mode}
		}{
			\toggletrue{LCart}
			\togglefalse{LCpres}
			\togglefalse{LCposter}
			\wlog{Article mode}
		}
	}
	\makeatother%

%Language options ([french, english]) should be on the document level (last is main); except with tikzposter: put [french, english] options next to \usepackage{babel} to avoid warning. beamer uses the \translate command for the appendix: omitting babel results in a warning, see https://github.com/josephwright/beamer/issues/449. Babel also seems required for \refname.
%TODO poster!
	\iftoggle{LCpres}{
		\usepackage{babel}
	}{
	}
	%\frenchbsetup{AutoSpacePunctuation=false}
%listings (1.7) does not allow multi-byte encodings. listingsutf8 works around this only for characters that can be represented in a known one-byte encoding and only for \lstinputlisting. Other workarounds: use literate mechanism; or escape to LaTeX (but breaks alignment).
	%\usepackage{listings}
	%\lstset{tabsize=2, basicstyle=\ttfamily, escapechar=§, literate={é}{{\'e}}1}
%I favor acro over acronym because the former is more recently updated (2018 VS 2015 at time of writing); has a longer user manual (about 40 pages VS 6 pages if not counting the example and implementation parts); has a command for capitalization; and acronym suffers a nasty bug when ac used in section, see https://tex.stackexchange.com/q/103483 (though this might be the fault of the silence package and might be solved in more recent versions, I do not know) and from a bug when used with cleveref, see https://tex.stackexchange.com/q/71364. However, loading it makes compilation time (one pass on this template) go from 0.6 to 1.4 seconds, see https://bitbucket.org/cgnieder/acro/issues/115. Option short-format not usable in the package options as it is fragile, see https://tex.stackexchange.com/q/466882.
	%\usepackage[single]{acro}
	%\acsetup{short-format = {\scshape}}
	%\DeclareAcronym{AMCD}{short=amcd, long={Aide Multicritère à la Décision}}
\DeclareAcronym{AR}{short=ar, long={Argumentative Recommender}}
\DeclareAcronym{DA}{short=da, long={Decision Analysis}}
\DeclareAcronym{DJ}{short=dj, long={Deliberated Judgment}}
\DeclareAcronym{DM}{short=dm, long={Decision Maker}}
\DeclareAcronym{DP}{short=dp, long={Deliberated Preference}}
\DeclareAcronym{MAVT}{short=mavt, long={Multiple Attribute Value Theory}}
\DeclareAcronym{MCDA}{short=mcda, long={Multicriteria Decision Aid}}
\DeclareAcronym{MIP}{short=mip, long={Mixed Integer Program}}


\iftoggle{LCpres}{
	%I favor fmtcount over nth because it is loaded by datetime anyway; and fmtcount warns about possible conflicts when loaded after nth.
	\usepackage{fmtcount}
	%For nice input of date of presentation. Must be loaded after the babel package. Has possible problems with srcletter: https://golatex.de/verwendung-von-babel-und-datetime-in-scrlttr2-schlaegt-fehlt-t14779.html.
	\usepackage[nodayofweek]{datetime}
}{
}
%For presentations, Beamer implicitely uses the pdfusetitle option. ntheorem doc says to load hyperref “before the first use of \newtheorem”. autonum doc mandates option hypertexnames=false. I want to highlight links only if necessary for the reader to recognize it as a link, to reduce distraction. In presentations, this is already taken care of by beamer. If using colorlinks=true in a presentation, see https://tex.stackexchange.com/q/203056. Crashes the first compilation with tikzposter, just compile again and the problem disappears, see https://tex.stackexchange.com/q/254257.
\makeatletter
\iftoggle{LCpres}{
	\usepackage{hyperref}
}{
	\usepackage[hypertexnames=false, pdfusetitle, linkbordercolor={1 1 1}, citebordercolor={1 1 1}, urlbordercolor={1 1 1}]{hyperref}
	%https://tex.stackexchange.com/a/466235
	\pdfstringdefDisableCommands{%
		\let\thanks\@gobble
	}
}
\makeatother
%urlbordercolor is used both for \url and \doi, which I think shouldn’t be colored, and for \href, thus might want to color manually when required. Requires xcolor.
	\newcommand{\hrefblue}[2]{\textcolor{blue}{\href{#1}{#2}}}
%hyperref doc says: “Package bookmark replaces hyperref’s bookmark organization by a new algorithm (...) Therefore I recommend using this package”.
	\usepackage{bookmark}
%Need to invoke hyperref explicitly to link to line numbers: \hyperlink{lintarget:mylinelabel}{\ref*{lin:mylinelabel}}, with \ref* to disable automatic link. Also see https://tex.stackexchange.com/q/428656 for referencing lines from another document.
	%\usepackage{lineno}
	%\newcommand{\llabel}[1]{\hypertarget{lintarget:#1}{}\linelabel{lin:#1}}
	%\setlength\linenumbersep{9mm}
%For complex authors blocks. Seems like authblk wants to be later than hyperref, but sooner than silence. See https://tex.stackexchange.com/q/475513 for the patch to hyperref pdfauthor.
	\ExplSyntaxOn
	\seq_new:N \g_oc_hrauthor_seq
	\NewDocumentCommand{\addhrauthor}{m}{
		\seq_gput_right:Nn \g_oc_hrauthor_seq { #1 }
	}
	%Should be \NewExpandableDocumentCommand, but this is not yet provided by my version of xparse
	\DeclareExpandableDocumentCommand{\hrauthor}{}{
		\seq_use:Nn \g_oc_hrauthor_seq {,~}
	}
	\ExplSyntaxOff
	{
		\catcode`#=11\relax
		\gdef\fixauthor{\xpretocmd{\author}{\addhrauthor{#2}}{}{}}%
	}
	\iftoggle{LCart}{
		\usepackage{authblk}
		\renewcommand\Affilfont{\small}
		\fixauthor
		\AtBeginDocument{
		    \hypersetup{pdfauthor={\hrauthor}}
		}
	}{
	}
%I do not use floatrow, because it requires an ugly hack for proper functioning with KOMA script (see scrhack doc). Instead, the following command centers all floats (using \centering, as the center environment adds space, http://texblog.net/latex-archive/layout/center-centering/), and I manually place my table captions above and figure captions below their contents (https://tex.stackexchange.com/a/3253).
	\makeatletter
	\g@addto@macro\@floatboxreset\centering
	\makeatother
%Permits to customize enumeration display and references
	%\nottoggle{LCpres}{
		%\usepackage{enumitem} %follow list environments by a string to customize enumeration, example: \begin{description}[itemindent=8em, labelwidth=!] or \begin{enumerate}[label=({\roman*}), ref={\roman*}].
	%}{
	%}
%Provides \Cen­ter­ing, \RaggedLeft, and \RaggedRight and en­vi­ron­ments Cen­ter, FlushLeft, and FlushRight, which al­low hy­phen­ation. With tikzposter, seems to cause 1=1 to be printed in the middle of the poster.
	%\usepackage{ragged2e}
%To typeset units by closely following the “official” rules.
	%\usepackage[strict]{siunitx}
%Turns the doi provided by some bibliography styles into URLs. However, uses old-style dx.doi url (see 3.8 DOI system Proxy Server technical details, “Users may resolve DOI names that are structured to use the DOI system Proxy Server (https://doi.org (current, preferred) or earlier syntax http://dx.doi.org).”, https://www.doi.org/doi_handbook/3_Resolution.html). The patch solves this.
	\usepackage{doi}
	\makeatletter
	\patchcmd{\@doi}{http://dx.doi.org}{https://doi.org}{}{}
	\makeatother
%Makes sure upper case greek letters are italic as well.
	\usepackage{fixmath}
%Provides \mathbb; obsoletes latexsym (see http://tug.ctan.org/macros/latex/base/latexsym.dtx). Relatedly, \usepackage{eucal} to change the mathcal font and \usepackage[mathscr]{eucal} (apparently equivalent to \usepackage[mathscr]{euscript}) to supplement \mathcal with \mathscr. This last option is not very useful as both fonts are similar, and the intent of the authors of eucal was to provide a replacement to mathcal (see doc euscript). Also provides \mathfrak for supplementary letters.
	\usepackage{amsfonts}
%Provides a beautiful (IMHO) \mathscr and really different than \mathcal, for supplementary uppercase letters. But there is no bold version. Alternative: mathrsfs (more slanted), but when used with tikzposter, it warns about size substitution, see https://tex.stackexchange.com/q/495167.
	\usepackage[scr]{rsfso}
%Multiple means to produce bold math: \mathbf, \boldmath (defined to be \mathversion{bold}, see fntguide), \pmb, \boldsymbol (all legacy, from LaTeX base and AMS), \bm (the most recommended one), \mathbold from package fixmath (I don’t see its advantage over \boldsymbol).
%“The \boldsymbol command is obtained preferably by using the bm package, which provides a newer, more powerful version than the one provided by the amsmath package. Generally speaking, it is ill-advised to apply \boldsymbol to more than one symbol at a time.” — AMS Short math guide. “If no bold font appears to be available for a particular symbol, \bm will use ‘poor man’s bold’” — bm. It is “best to load the package after any packages that define new symbol fonts” – bm. bm defines \boldsymbol as synonym to \bm. \boldmath accesses the correct font if it exists; it is used by \bm when appropriate. See https://tex.stackexchange.com/a/10643 and https://github.com/latex3/latex2e/issues/71 for some difficulties with \bm.
	\usepackage{bm}
	\nottoggle{LCpres}{
	%https://ctan.org/pkg/amsmath recommends ntheorem, which supersedes amsthm, which corrects the spacing of proclamations and allows for theoremstyle. Option standard loads amssymb and latexsym. Must be loaded after amsmath (from ntheorem doc). From cleveref doc, “ntheorem is fully supported and even recommended”; says to load cleveref after ntheorem. When used with tikzposter, warns about size substitution for the lasy (latexsym) font when using \url, because ntheorem loads latexsym; relatedly (but not directly related to ntheorem), size substitution warning with the cmex font happens when loading amsmath and using \url.%TODO poster
		\usepackage[thmmarks, amsmath, standard, hyperref]{ntheorem}
		%empheq doc says to do this after loading ntheorem
		\usetagform{default}
	%Provides \cref. Unfortunately, cref fails when the language is French and referring to a label whose name contains a colon (https://tex.stackexchange.com/q/83798). Use \cref{sec\string:intro} to work around this. cleveref should go “laster” than hyperref.
		\usepackage{cleveref}
	}{
	}
	\iftoggle{LCart}{
	%Equations get numbers iff they are referenced. Loading order should be “amsmath → hyperref → cleveref → autonum”, according to autonum doc. Use this in preference to the showonlyrefs option from mathtools, see https://tex.stackexchange.com/q/459918 and autonum doc. See https://tex.stackexchange.com/a/285953 for the etex line. Incompatible with my version of tikzposter (produces “! Improper \prevdepth”).
		\expandafter\def\csname ver@etex.sty\endcsname{3000/12/31}\let\globcount\newcount
		\usepackage{autonum}
	}{
	}
%Also loaded by tikz.
	\usepackage{xcolor}
\iftoggle{LCpres}{
	\usepackage{tikz}
	%\usetikzlibrary{babel, matrix, fit, plotmarks, calc, trees, shapes.geometric, positioning, plothandlers, arrows, shapes.multipart}
}{
}
%Vizualization, on top of TikZ
	%\usepackage{pgfplots}
	%\pgfplotsset{compat=1.14}
\usepackage{graphicx}
	\graphicspath{{graphics/}}

%Provides \print­length{length}, useful for debugging.
	%\usepackage{printlen}
	%\uselengthunit{mm}

\iftoggle{LCpres}{
	\usepackage{appendixnumberbeamer}
	%I have yet to see anyone actually use these navigation symbols; let’s disable them
	\setbeamertemplate{navigation symbols}{} 
	\usepackage{preamble/beamerthemeParisFrance}
	\setcounter{tocdepth}{10}
}{
}

%Do not use the displaymath environment: use equation. Do not use the eqnarray or eqnarray* environments: use align(*). This improves spacing. (See l2tabu or amsldoc.)


%Requires package xcolor.
\newcommand{\commentOC}[1]{\textcolor{blue}{\small$\big[$OC: #1$\big]$}}
%Requires package babel and option [french]. According to babel doc, need two braces around \selectlanguage to make the changes really local.
\newcommand{\commentOCf}[1]{\textcolor{blue}{{\small\selectlanguage{french}$\big[$OC : #1$\big]$}}}
\newcommand{\commentYM}[1]{\textcolor{red}{\small$\big[$YM: #1$\big]$}}
\newcommand{\commentYMf}[1]{\textcolor{red}{{\small\selectlanguage{french}$\big[$YM : #1$\big]$}}}

\bibliographystyle{abbrvnat}
\newcommand{\possessivecite}[1]{\citeauthor{#1}’s \citeyearpar{#1}}

%https://tex.stackexchange.com/a/467188, https://tex.stackexchange.com/a/36088 - uncomment if one of those symbols is used.
%\DeclareFontFamily{U} {MnSymbolD}{}
%\DeclareFontShape{U}{MnSymbolD}{m}{n}{
%  <-6> MnSymbolD5
%  <6-7> MnSymbolD6
%  <7-8> MnSymbolD7
%  <8-9> MnSymbolD8
%  <9-10> MnSymbolD9
%  <10-12> MnSymbolD10
%  <12-> MnSymbolD12}{}
%\DeclareFontShape{U}{MnSymbolD}{b}{n}{
%  <-6> MnSymbolD-Bold5
%  <6-7> MnSymbolD-Bold6
%  <7-8> MnSymbolD-Bold7
%  <8-9> MnSymbolD-Bold8
%  <9-10> MnSymbolD-Bold9
%  <10-12> MnSymbolD-Bold10
%  <12-> MnSymbolD-Bold12}{}
%\DeclareSymbolFont{MnSyD} {U} {MnSymbolD}{m}{n}
%\DeclareMathSymbol{\ntriplesim}{\mathrel}{MnSyD}{126}
%\DeclareMathSymbol{\nlessgtr}{\mathrel}{MnSyD}{192}
%\DeclareMathSymbol{\ngtrless}{\mathrel}{MnSyD}{193}
%\DeclareMathSymbol{\nlesseqgtr}{\mathrel}{MnSyD}{194}
%\DeclareMathSymbol{\ngtreqless}{\mathrel}{MnSyD}{195}
%\DeclareMathSymbol{\nlesseqgtrslant}{\mathrel}{MnSyD}{198}
%\DeclareMathSymbol{\ngtreqlessslant}{\mathrel}{MnSyD}{199}
%\DeclareMathSymbol{\npreccurlyeq}{\mathrel}{MnSyD}{228}
%\DeclareMathSymbol{\nsucccurlyeq}{\mathrel}{MnSyD}{229}
%\DeclareFontFamily{U} {MnSymbolA}{}
%\DeclareFontShape{U}{MnSymbolA}{m}{n}{
%  <-6> MnSymbolA5
%  <6-7> MnSymbolA6
%  <7-8> MnSymbolA7
%  <8-9> MnSymbolA8
%  <9-10> MnSymbolA9
%  <10-12> MnSymbolA10
%  <12-> MnSymbolA12}{}
%\DeclareFontShape{U}{MnSymbolA}{b}{n}{
%  <-6> MnSymbolA-Bold5
%  <6-7> MnSymbolA-Bold6
%  <7-8> MnSymbolA-Bold7
%  <8-9> MnSymbolA-Bold8
%  <9-10> MnSymbolA-Bold9
%  <10-12> MnSymbolA-Bold10
%  <12-> MnSymbolA-Bold12}{}
%\DeclareSymbolFont{MnSyA} {U} {MnSymbolA}{m}{n}
%%Rightwards wave arrow: ↝. Alternative: \rightsquigarrow from amssymb, but it’s uglier
%\DeclareMathSymbol{\rightlsquigarrow}{\mathrel}{MnSyA}{160}

%03B3 Greek Small Letter Gamma
\newunicodechar{γ}{\gamma}
%03B4 Greek Small Letter Delta
\newunicodechar{δ}{\delta}
%2115 Double-Struck Capital N
\newunicodechar{ℕ}{\mathbb{N}}
%211D Double-Struck Capital R
\newunicodechar{ℝ}{\mathbb{R}}
%21CF Rightwards Double Arrow with Stroke
\newunicodechar{⇏}{\nRightarrow}
%21D2 Rightwards Double Arrow
\newunicodechar{⇒}{\ensuremath{\Rightarrow}}
%21D4 Left Right Double Arrow
\newunicodechar{⇔}{\Leftrightarrow}
%21DD Rightwards Squiggle Arrow
\newunicodechar{⇝}{\rightsquigarrow}
%2212 Minus Sign
\newunicodechar{−}{\ifmmode{-}\else\textminus\fi}
%2227 Logical And
\newunicodechar{∧}{\land}
%2228 Logical Or
\newunicodechar{∨}{\lor}
%2229 Intersection
\newunicodechar{∩}{\cap}
%222A Union
\newunicodechar{∪}{\cup}
%2260 Not Equal To (handy also as text in informal writing)
\newunicodechar{≠}{\ensuremath{\neq}}
%2264 Less-Than or Equal To
\newunicodechar{≤}{\leq}
%2265 Greater-Than or Equal To
\newunicodechar{≥}{\geq}
%2270 Neither Less-Than nor Equal To
\newunicodechar{≰}{\nleq}
%2271 Neither Greater-Than nor Equal To
\newunicodechar{≱}{\ngeq}
%2272 Less-Than or Equivalent To
\newunicodechar{≲}{\lesssim}
%2273 Greater-Than or Equivalent To
\newunicodechar{≳}{\gtrsim}
%2274 Neither Less-Than nor Equivalent To – also, from MnSymbol: \nprecsim, a more exact match to the Unicode symbol; and \npreccurlyeq, too small
\newunicodechar{≴}{\not\preccurlyeq}
%2275 Neither Greater-Than nor Equivalent To
\newunicodechar{≵}{\not\succcurlyeq}
%2279 Neither Greater-Than nor Less-Than – requires MnSymbol; also \nlessgtr from txfonts/pxfonts, \ngtreqless from MnSymbol (but much higher), \ngtrless from MnSymbol (a more exact match to the Unicode symbol); for incomparability (not matching this Unicode symbol), may also consider \ntriplesim from MnSymbol,\nparallelslant from fourier, \between from mathabx, or ⋈
\newunicodechar{≹}{\ngtreqlessslant}
%227A Precedes
\newunicodechar{≺}{\prec}
%227B Succeeds
\newunicodechar{≻}{\succ}
%227C Precedes or Equal To
\newunicodechar{≼}{\preccurlyeq}
%227D Succeeds or Equal To
\newunicodechar{≽}{\succcurlyeq}
%227E Precedes or Equivalent To
\newunicodechar{≾}{\precsim}
%227F Succeeds or Equivalent To
\newunicodechar{≿}{\succsim}
%2280 Does Not Precede
\newunicodechar{⊀}{\nprec}
%2281 Does Not Succeed
\newunicodechar{⊁}{\nsucc}
%22B2 Normal Subgroup Of – \triangleleft is too small compared to \trianglelefteq and the like; \lhd seems equivalent to \vartriangleleft from amssymb
\newunicodechar{⊲}{\vartriangleleft}
%22B3 Contains as Normal Subgroup (also: 25B7 White right-pointing triangle or 25B9 White right-pointing small triangle)
\newunicodechar{⊳}{\vartriangleright}
%22B4 Normal Subgroup of or Equal To
\newunicodechar{⊴}{\trianglelefteq}
%22B5 Contains as Normal Subgroup or Equal To
\newunicodechar{⊵}{\trianglerighteq}
%22C8 Bowtie
\newunicodechar{⋈}{\bowtie}
%22EA Not Normal Subgroup Of
\newunicodechar{⋪}{\ntriangleleft}
%22EB Does Not Contain As Normal Subgroup – TODO say that it requires amssymb (or amsfont?); check why \triangleright is used in formal instead of \rhd.
\newunicodechar{⋫}{\ntriangleright}
%22EC Not Normal Subgroup of or Equal To
\newunicodechar{⋬}{\ntrianglelefteq}
%22ED Does Not Contain as Normal Subgroup or Equal
\newunicodechar{⋭}{\ntrianglerighteq}
%25A1 White Square
\newunicodechar{□}{\Box}
%27E6 Mathematical Left White Square Bracket – there’s also \llbracket from stmaryrd
\newunicodechar{⟦}{\text{\textlbrackdbl}}
%27E7 Mathematical Right White Square Bracket – there’s also \rrbracket from stmaryrd
\newunicodechar{⟧}{\text{\textrbrackdbl}}
%27FC Long Rightwards Arrow from Bar
\newunicodechar{⟼}{\longmapsto}
%2AB0 Succeeds Above Single-Line Equals Sign
\newunicodechar{⪰}{\succeq}
%301A Left White Square Bracket
\newunicodechar{〚}{\textlbrackdbl}
%301B Right White Square Bracket
\newunicodechar{〛}{\textrbrackdbl}
%→ is defined by default as \textrightarrow, which is invalid in math mode. Same thing for the three other commands. I redefine those four using \DeclareUnicodeCharacter instead of \newunicodechar because the latter warns about the previous definition.
%→ Rightwards Arrow
\DeclareUnicodeCharacter{2192}{\ifmmode\rightarrow\else\textrightarrow\fi}
%¬ Not Sign
\DeclareUnicodeCharacter{00AC}{\ifmmode\lnot\else\textlnot\fi}
%… Horizontal Ellipsis
\DeclareUnicodeCharacter{2026}{\ifmmode\dots\else\textellipsis\fi}
%× Multiplication Sign
\DeclareUnicodeCharacter{00D7}{\ifmmode\times\else\texttimes\fi}


\newcommand{\R}{ℝ}
\newcommand{\N}{ℕ}
%\mathscr is rounder than \mathcal.
\newcommand{\powerset}[1]{\mathscr{P}(#1)}
%Powerset without zero.
\newcommand{\powersetz}[1]{\mathscr{P}^*(#1)}
%https://tex.stackexchange.com/a/45732, works within both \set and \set*, same spacing than \mid (https://tex.stackexchange.com/a/52905).
\newcommand{\suchthat}{\;\ifnum\currentgrouptype=16 \middle\fi|\;}
%Integer interval.
\newcommand{\intvl}[1]{⟦#1⟧}
%Allows for \abs and \abs*, which resizes the delimiters.
\DeclarePairedDelimiter\abs{\lvert}{\rvert}
\DeclarePairedDelimiter\card{\lvert}{\rvert}
%Perhaps should use U+2016 ‖ DOUBLE VERTICAL LINE here?
\DeclarePairedDelimiter\norm{\lVert}{\rVert}
%Better than using the package braket because braket introduces possibly undesirable space. Then: \begin{equation}\set*{x \in \R^2 \suchthat \norm{x}<5}\end{equation}.
\DeclarePairedDelimiter\set{\{}{\}}
\DeclareMathOperator*{\argmax}{arg\,max}
\DeclareMathOperator*{\argmin}{arg\,min}

%We want the straight form of \phi for mathematics, as recommended in UTR #25: Unicode support for mathematics, and thus use \phi for the mathematical symbol and not \varphi; and similarly \epsilon is preferred to \varepsilon for the mathematical symbol.

%The amssymb solution.
%\newcommand{\restr}[2]{{#1}_{\restriction #2}}
%Another acceptable solution.
%\newcommand{\restr}[2]{{#1|}_{#2}}
%https://tex.stackexchange.com/a/278631; drawback being that sometimes the text collides with the line below.
\newcommand\restr[2]{#1\raisebox{-.5ex}{$|$}_{#2}}


%Decision Theory (MCDA and SC)
\newcommand{\allalts}{\mathscr{A}}
\newcommand{\allcrits}{\mathscr{C}}
\newcommand{\alts}{A}
\newcommand{\dm}{i}
\newcommand{\allF}{\mathscr{F}}
\newcommand{\allvoters}{\mathscr{N}}
\newcommand{\voters}{N}
\newcommand{\allprofs}{\boldsymbol{\mathcal{R}}}
\newcommand{\prof}{\boldsymbol{R}}
\newcommand{\linors}{\mathscr{L}(\allalts)}
%Thanks to https://tex.stackexchange.com/q/154549
	%\makeatletter
	%\def\@myRgood@#1#2{\mathrel{R^X_{#2}}}
	%\def\myRgood{\@ifnextchar_{\@myRgood@}{\mathrel{R^X}}}
	%\makeatother

%Deliberated Judgment
\newcommand{\allargs}{S^*}
\newcommand{\args}{S}
\newcommand{\ar}{s}
\newcommand{\ileadsto}{⇝}
\newcommand{\ibeatse}{⊳_\exists}
\newcommand{\nibeatse}{⋫_\exists}
\newcommand{\ibeatsst}{⊳_\forall}
\newcommand{\nibeatsst}{⋫_\forall}
\newcommand{\mleadsto}[1][\eta]{⇝_{#1}}
\newcommand{\mbeats}[1][\eta]{⊳_{#1}}
\newcommand{\ibeatseinv}{⊳_\exists^{-1}}

%Logic
\newcommand{\ltru}{\texttt{T}}
\newcommand{\lfal}{\texttt{F}}


%\DeclareAcronym{AMCD}{short=amcd, long={Aide Multicritère à la Décision}}
\DeclareAcronym{AR}{short=ar, long={Argumentative Recommender}}
\DeclareAcronym{DA}{short=da, long={Decision Analysis}}
\DeclareAcronym{DJ}{short=dj, long={Deliberated Judgment}}
\DeclareAcronym{DM}{short=dm, long={Decision Maker}}
\DeclareAcronym{DP}{short=dp, long={Deliberated Preference}}
\DeclareAcronym{MAVT}{short=mavt, long={Multiple Attribute Value Theory}}
\DeclareAcronym{MCDA}{short=mcda, long={Multicriteria Decision Aid}}
\DeclareAcronym{MIP}{short=mip, long={Mixed Integer Program}}



%I find these settings useful in draft mode. Should be removed for final versions.
	%Which line breaks are chosen: accept worse lines, therefore reducing risk of overfull lines. Default = 200.
		\tolerance=2000
	%Accept overfull hbox up to...
		\hfuzz=2cm
	%Reduces verbosity about the bad line breaks.
		\hbadness 5000
	%Reduces verbosity about the underful vboxes.
		\vbadness=1300

\title{Deliberated diet}
\author{Olivier Cailloux}
\author{Yves Meinard}
\affil{Université Paris-Dauphine, PSL Research University, CNRS, LAMSADE, 75016 PARIS, FRANCE}
\author{Nicolas Salliou}
\affil{Affiliation}
\hypersetup{
	pdfsubject={},
	pdfkeywords={},
}
\begin{document}
\maketitle

\section{Context}
This section presents the context of the study.
This is a draft proposal open for discussion.
 
\subsection{Goal}
We aim at studying how we can capture the deliberated judgment of an individual towards a complex, practical topic. The deliberated judgment of an individual towards a topic is the stance she adopts after careful examination of all relevant arguments concerning that topic.

In this exploratory study, we want to try to capture deliberated judgments on a particular question, of many individuals, by confronting them to many arguments, and extract from this lessons about how to capture deliberated judgments more generally, and about the difficulties that such a process gets confronted to. 

We proceed as follows. We choose a practical question as our topic. We select two champions, that is, two persons that know the arguments for and against particular stances on that question, and that have opposed positions on which stance should be the deliberated one. In the \emph{collection} phase, the two champions defend their respective positions by recording their arguments for their position and against other positions in videos. In the \emph{adjudication} phase, we show these videos to individuals, following a well-defined protocol, through a web site. The individual is called a visitor (of the web site). The visitor is questioned during the process of viewing the arguments and counter-arguments displayed in the videos, about the evolution of her judgment, among others.

\subsection{Topic}
\label{sec:topic}
We are interested in capturing the deliberated judgment of individuals concerning a practical question in a fictitious setting. Here is the question we chose.

In a fictitious place, a local authority considers building a new public cantine. What kind of menu should the cantine offer to its customers?
\begin{enumerate}
	\item \label{it:vegan} It should propose only vegan food every day
	\item \label{it:four} It should propose only vegan food on four days a week and vegan and non-vegan choice on the remaining day
	\item \label{it:one} It should propose only vegan food on one day per week and vegan and non-vegan choice on the remaining days
	\item \label{it:zero} It should propose vegan and non-vegan choice every day
\end{enumerate}

From now on, assume we have two champions, $C_a$ and $C_b$.

\section{Collection phase}
We ask $C_a$ and $C_b$ for their favorite propositions, defined as $t_a$ and $t_b$ respectively (to choose among \cref{it:vegan,it:four,it:one,it:zero} as defined in \cref{sec:topic}). Hopefully, we have chosen $C_a$ and $C_b$ so that $t_a$ and $t_b$ are at opposite ends of the spectrum.

We ask $C_a$ to produce a video in favor of $t_a$, and label it $s_{1a}$. We send $s_{1a}$ to $C_b$. We give $C_b$ a choice: either produce a reply video (arguing for $t_b$ against the video arguing for $t_a$), or start a new thread in favor of $t_b$, or both. In the first case, we label the argument $s_{1ab}$. In the second case, we label the argument $s_{1b}$. The scheme is then repeated. We also run a parallel scheme starting with $b$ instead of $a$.

Here is a more precise description of the collection phase.

\begin{itemize}
	\item At the start of the whole procedure, both champions are fully informed about the procedure and the future use of their arguments, including about the time constraints (effectively as if they received a copy of this whole document, except possibly with a different form more suitable for easy understanding of their role).
	\item The naming scheme is such that the last letter of a video indicate its author.
	\item Define $\text{init}(\alpha)$, with $\alpha \in \set{a, b}$, as follows. Precondition: $C_\alpha$ has received no argument of any sort from her opponent. We ask $C_\alpha$ to produce a video in favor of $t_\alpha$, and label the resulting video $s_{1 \alpha}$.
	\item Define $\text{next}(\alpha)$, with $\alpha \in \set{a, b}$, as one plus the greatest integer numbering a start video recorded by $\alpha$ in the whole process so far. For example, if $\alpha = b$, and if a video labeled $s_{3b}$ exists already but no video labeled $s_{4b}$ exists yet, then $\text{next}(b) = 4$.
	\item Define $\text{reply}(S)$, with $S$ a set of previously produced videos by a given champion $\alpha$, as follows. We address the champion $\beta$, with $\beta ≠ \alpha$. We send her the videos in $S$. We invite her to reply to every videos she wants to reply to by recording new videos, in about two weeks time. A video replying to another video is labeled by suffixing $\beta$ to the label of the video it replies to. For example, if $\alpha = a$, when replying to a video labeled $s_{1aba}$, we label it $s_{1abab}$. The set of replies may be empty. She may also start a new video that is not a reply, that we name $s_{k\beta}$, with $k$ equal to $\text{next}(\beta)$.
	\item We start with $\text{init}(a)$ and $\text{init}(b)$ in parallel, therefore obtaining $s_{1a}$ and $s_{1b}$. Define $S_{1a} = \set{s_{1a}}$ and $S_{1b} = \set{s_{1b}}$.
	\item After having obtained a set of videos $S$, if $S ≠ \emptyset$, we run $\text{reply}(S)$, and label the set of resulting videos by suffixing the identifier of their author to the label of $S$. 
	\item We repeat the previous step, running two threads in parallel whenever possible, until both participants think that all arguments that is important to form a deliberated judgment has been given.
	\item When both participants have finished producing videos, we ask them to label briefly (in maximum 30 characters) each of their own videos and select an image from the video that becomes its “thumbnail”.
\end{itemize}

\begin{example}
Here is an example run.
\begin{enumerate}
	\item We start with $\text{init}(a)$ and $\text{init}(b)$ in parallel and obtain $s_{1a}$ and $s_{1b}$. Define $S_{a} = \set{s_{1a}}$ and $S_{b} = \set{s_{1b}}$.
	\item We apply $\text{reply}(\set{s_{1a}})$ and obtain $S_{ab} = \set{s_{1ab}, s_{2b}}$.
	\item In parallel, we apply $\text{reply}(\set{s_{1b}})$ and obtain $S_{ba} = \set{s_{1ba}}$.
	\item We apply $\text{reply}(S_{ab})$ (as soon as $S_{ab}$ is available) and obtain $S_{aba} = \set{s_{1aba}, s_{2ba}, s_{2a}}$.
	\item We apply $\text{reply}(S_{ba})$ (as soon as $S_{ba}$ is available), but $C_b$ sees no need to answer those arguments; we obtain $S_{bab} = \emptyset$.
	\item We apply $\text{reply}(S_{aba})$ (as soon as $S_{aba}$ is available), where $C_b$ sees an opportunity for answering; we obtain $S_{abab} = \set{s_{2bab}}$.
	\item We apply $\text{reply}(S_{abab})$ (as soon as the argument is available), $C_a$ sees no need of counter-arguing, and we obtain $S_{ababa} = \emptyset$.
\end{enumerate}
\end{example}

\section{Adjudication phase}
When the collection phase is over, we put every collected video on a web site and design everything so that the adjudication phase of the protocol can be run, as described in this section. This phase consists, for each visitor, in 

An individual $i$ comes to our web site. This visitor has to go through the following two sub-phases (and an optional third one): the \emph{context presentation} and the \emph{constrained deliberation} sub-phases; and can then optionally proceed to the \emph{free deliberation} sub-phase.

\subsection{Context presentation}
In the context presentation sub-phase, the visitor is explained the context of the study by written text. The following points should be made clear to the visitor.
\begin{itemize}
	\item We (the team designing the study) are interested in knowing what the visitor considers are the best arguments to form a deliberated judgment on this topic, as judged by himself. Our long-term goal is to help individuals form a well-informed opinion by displaying to them arguments that tend to be considered good by many individuals, taking into account only the quality of the argument and not on the stance that the argument supports.
	\item We have no conflict of interest and do not try to promote either diet option (contrary to our champions). \commentOC{It will strike the visitor that we seem biased towards veganism or vegetarianism: the choices are not symmetrical from the POV of a meat-lover; and no champion defend a meat-extensive diet. We should say something about this. What?}
	\item Two well-known public figures have argued for two specific options. Those champions, contrary to us, actively try to promote one of the options.
	\item The visitor is shown the question of the topic and the possible choices.
	\item He is asked whether he is willing to adopt an open mind, that is, to be ready to change his mind if given arguments that he considers good for a side that he does not consider a priori as the “right” choice. \commentOC{New proposal, to maximize our chance of observing something interesting.}
	\item He should spend one hour and a half, (mostly) uninterrupted, on this experiment. During this time, he will have to look at videos from both sides, with a balanced time for both champions, and answer questions about his current judgment and about which arguments he finds good \commentOC{This is a new proposition. The previous version was: invited to spend as much time as he wants watching videos (without restriction of balancing the time spent for each champion) to form an opinion.} \commentOC{I propose to drop the socio-demographic questions.} 
	\item After this hour and a half, he is free to continue exploring the videos and answering further questions in an unconstrained way.
	\item The visitor must agree that we use the collected data (RGPD).
\end{itemize}
Only after the visitor has accepted those conditions can he start the controlled deliberation phase. 

\subsection{Constrained deliberation}
During the constrained deliberation sub-phase, he can still access the explanations relating to the context presentation if he so desires, but it is not displayed prominently any more as we expect it will not be his main interest in that phase.

The constrained deliberation phase is composed of \textbf{video steps} and \textbf{questionaire steps}. 

\subsubsection{Video step}
Here is the description of a video step.
\begin{itemize}
	\item Each video step starts with an associated set of videos “proposed”, $S_P$, and an associated list of videos “seen”, $S_S$. The list $S_S$ is possibly augmented at the end of each step while the set $S_P$ is computed at the start of each step from the list $S_S$ resulting from the previous step. Two time counters $\theta_a$ and $\theta_b$ keep track of the time spent so far watching videos of each champion.
	\item At the start, $S_S = \emptyset$; $\theta_a = 0$; $\theta_b = 0$.
	\item A “video step” consists in the set $S_P$ being shown to the individual among which he can choose the video he will watch during this step; and he can also watch again videos that are “seen”. Each video is displayed as its thumbnail, with its label shown clearly, in a randomized order; the videos from $S_S$ are clearly distinguished and displayed afterwards, in the order they have entered the list (the order they have been marked as “seen”). He clicks on a video and starts watching it. 
He can navigate in the timeline of the video (a la youtube).
The step is finished with one of these possibilities.
	\begin{itemize}
		\item If the video is watched until the end with no navigation in the timeline, in which case the video is added at the end of $S_S$. 
		\item The user can also click to mark the video as seen without having watched it entirely, in which case the video is also added at the end of $S_S$.
		\item The user can also click to stop the video without marking it as seen; in which case $S_S$ is not modified and the video is added at the end of $S_T$. \commentOC{What’s that?}
	\end{itemize}
	\item Given a video $s$, define $r(s)$ as the singleton set containing the reply video to $s$ (thus produced by the other champion than the author of $s$), if such a video exists, and $\emptyset$ otherwise. For example, $r(s_{2aba}) = \set{s_{2abab}}$ if such a video exists. At the start of a step where the list of videos seen is $S_S$, $S_N$ (for “next”) is defined as the non-seen videos among the starting videos and the videos replying to a video that has been seen, thus, $S_N = \left(\set{s_{k\alpha}, k \in \N, \alpha \in \set{a, b}} \cup \bigcup_{s \in S_S} r(s)\right) \setminus S_S$. The proposed videos in this step are defined as $S_P = S_N$; \emph{except} if the time allowed to watch one of the champions has been exhausted. The visitor has 40 minutes max to spend for each champion. Thus, if $\theta_a ≥ 40 \text{ minutes}$, $S_P$ consists in the videos of champion $b$ among $S_N$.
	\item At any time (except when watching a video in full-screen), $i$ sees how much time he has spent watching videos of each champion.
	\item After 90 minutes have passed, $i$ is informed that the controlled deliberation phase is finished. \commentOC{Check how to count time. What if $i$ gets disconnected and comes back 10 minutes later?}
\end{itemize}

\subsubsection{Questionaire step}
A questionaire step is defined as asking the following questions to $i$. \commentOC{New proposition. Previously: At each step, $i$ may click “questionaire” to go to the questionaire part. In this part he is informed that we suggest he answers these questions only after he has formed a deliberated opinion, but that he can anyway go back to the video part and come back to the questionaire whenever he wants and change his answers. I also removed several questions.}
\commentOC{Decide how to alternate questionaire and video steps.}
\begin{itemize}
	\item Which answer would you choose if you had to choose now (referring to \cref{sec:topic})?
	\item Which videos did you personally find most helpful to form a deliberated judgement about this question?
	\item Which videos would you select for displaying to other users to help them form a deliberated judgment about this question, assuming individuals have only a short time for watching the videos?
	\item Are there some arguments that you think have not been used from either champions and should have? (free text answer)
	\item Which arguments do you think the other champion has not replied to convincingly? (free text answer) \commentOC{These two questions should be asked only at the end of the constrained deliberation phase?}
\end{itemize}
For each of the “which videos” questions, $i$ is shows the list of thumbnails of videos in $S_S$, followed by the thumnails of videos he has partially seen but not marked as seen, and can check any thumnail he wants, from both champions. \commentOC{Should (or can?) order the videos?}

\subsection{Free deliberation}
After the constrained deliberation sub-phase, after having answered any final questionaire, the visitor can watch more videos. We continue sending her to some questionaire step from time to time. This phase is similar to the constrained deliberation phase, but with no time limit.

\section{Analysis}
\commentOC{We should think already now about some of the analysis that we will want to do, in order to ensure that our protocol is appropriate and to list the requirements on indicators collected during the adjudication phase.}
\begin{itemize}
	\item Do visitors change mind?
	\item Do visitors tend to spend more time watching videos from the side they favor a priori? Does this switch when they change their mind? We count the total time spent watching videos from each champion. We count the time really spent playing a video, including replays when $i$ has watched several times the same (part of a) video, and thus not counting fully a video that has been partly watched, even if the video has been included in $S_S$ because of an explicit demand from $i$. This permits to count time allowed to each champion.
	\item \commentOC{Think about this one.} est-ce qu’on peut raisonnablement dire qu’on a capturé le DJ ? Par exemple, sa position est-elle stable face à des arguments puisés dans une BD ? On pourrait comparer l’affirmation de stabilité sans protocole, ou après le protocole. Ou on pourrait comparer notre protocole à un autre et voir lequel amène à une position stable (donc proche du DJ).
	\item Do people agree on which videos are helpful? Try to cluster people so that they agree on this inside a cluster. Is this usable to shorten the time to deliberate?
\end{itemize}

\section{Think}
\begin{enumerate}
	\item Should we grant the right to the champions to discard some of their videos at the end of the collection phase? Or promote some? (They might think that some of their videos are much better than others.)
	\item We have to make sure $i$ understands that he will be proposed the “replies” videos only once he has marked the video as “seen”.
\end{enumerate}

\begin{enumerate}
	\item Determine and validate a procedure to build CAC models?
\end{enumerate}
\end{document}

